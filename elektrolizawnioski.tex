\documentclass{article}
\usepackage[polish]{babel}
\usepackage[utf8]{inputenc}
\usepackage[T1]{fontenc}
\usepackage{graphicx}
\usepackage{anysize}
\usepackage{enumerate}
\usepackage{times}
\usepackage{amsmath}

\usepackage{natbib}
\usepackage{graphicx}
\setcounter{section}{4}
\begin{document}

\section{Wnioski}

Obliczone wartości równoważnika elektrochemicznego, stałej Faradaya i ładunku elementarnego są zgodne z wartościami tablicowymi, ponieważ mieszczą się w przedziale niepewności. Niepewność względna jest wystarczająco mała aby uznać metodę wyznaczania stałej Faradaya i równoważnika elektrochemicznego za dokładną. Różnica między zmianą masy anod i katody może wynikać z przemywania lub niedokładnego wysuszenia, dlatego zwiększona została niepewność pomiarowa masy. Z otrzymanych pomiarów wynika że zmiana masy anod jest równa w granicach niepewności rozszerzonej  zmianie masy katody dzięki czemu można sformułować prawo zachowania masy.
\newpage

\setcounter{section}{4}
\section{Wnioski}

Obliczone wartości równoważnika elektrochemicznego, stałej Faradaya i ładunku elementarnego są zgodne z wartościami tablicowymi, ponieważ mieszczą się w przedziale niepewności. Niepewność względna jest wystarczająco mała aby uznać metodę wyznaczania stałej Faradaya i równoważnika elektrochemicznego za dokładną. Różnica między zmianą masy anod i katody może wynikać z przemywania lub niedokładnego wysuszenia, dlatego zwiększona została niepewność pomiarowa masy. Z otrzymanych pomiarów wynika że zmiana masy anod nie jest równa w granicach niepewności rozszerzonej zmianie masy katody więc nie można sformułować prawo zachowania masy.
\end{document}
