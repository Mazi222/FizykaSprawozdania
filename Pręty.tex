\documentclass{article}

\usepackage{polski}
\usepackage[utf8]{inputenc}
\usepackage{graphicx}
\usepackage{float}
\usepackage[margin=1in]{geometry}
\usepackage{graphicx}
\usepackage{amsmath}
\usepackage{mathtools}
\usepackage{amssymb}
\usepackage{multirow}
\usepackage{changepage}
\usepackage[export]{adjustbox}
\usepackage{wrapfig}
\usepackage{caption}

\captionsetup[figure]{name=}


\title{Sprawozdanie}
\begin{document}
	
	\begin{center}
		\bgroup
		\def\arraystretch{1.5}
		\begin{tabular}{|c|c|c|c|c|c|}
			\hline
			EAIiIB & \multicolumn{2}{|c|}{\begin{tabular}{@{}c@{}}Autor 1: Rafał Mazur \\Autor 2: Jakub Ficoń\end{tabular}} & Rok II & Grupa 5 & Zespół 3 \\
			\hline
			\multicolumn{3}{|c|}{\begin{tabular}{c}Temat: \\ Fale podłużne w ciałach stałych \end{tabular}} & 
			\multicolumn{3}{|c|}{\begin{tabular}{c}Numer ćwiczenia: \\ 29 \end{tabular}} \\
			\hline
			Data wykonania & Data oddania & Zwrot do poprawki & Data oddania & Data zaliczenia & Ocena \\[8ex]
			\hline
		\end{tabular}
		\egroup
	\end{center}  
	
	%WSTEP
	\section{Cel ćwiczenia}
	Wyznaczenie modułu Younga dla różnych materiałów na podstawie pomiaru prędkości rozchodzenia się fali dźwiękowej w pręcie.
	
	\section{Wstęp teoretyczny}
	    W załącznikach na końcu sprawozdania
	
	\section{Aparatura pomiarowa}
	    \begin{enumerate}
	        \item Waga elektroniczna o dokładności 1g
	        \item Zestaw 6 próbek służących do zmierzenia gęstości
	        \item Suwmiarka o dokładności do 0.05mm (0.00005m)
	        \item Miarka w rolce o podziałce 1mm
	        \item Zestaw 6 prętów o różnych kształtach
	        \item Młotek
	        \item Komputer stacjonarny z mikrofonem
	        \item Program Zelscope
	    \end{enumerate}
	
	\section{Wykonanie ćwiczenia}
	\begin{enumerate}
		\item Pomiar wymiarów próbek materiałów z których wykonane są pręty
		\item Zważenie próbek materiałów i na podstawie wymiarów i wagi wyliczenie gęstości
		\item Zmierzenie częstotliwości drgań harmonicznych dla prętów, przy pomocy programu Zelscope i mikrofonu ustawionego przy pręcie przez uderzeniu w pręt młotkiem.
		\item Powtórzenie powyższej czynności dla wszystkich prętów
	\end{enumerate}
	
	\pagebreak
	%KONIEC WSTEPU	
	\section{Wyniki pomiarów}
	\begin{figure}[!ht]
		\begin{adjustwidth}{-1cm}{}
			\def\arraystretch{1.3}
			\centering
			\begin{tabular}{|c|c|c|c|c|c|}
				\hline
				\parbox[c]{2cm}{\raggedright Materiał} & \begin{tabular}{c} Masa pręta\\ \mbox{[kg]}\end{tabular} & \begin{tabular}{c} Długość próbki  \\ \mbox{[cm]}  \end{tabular}  & 
				\begin{tabular}{c} Pole podstawy\\ \mbox{[$cm^2$]}
				\end{tabular}& 
				\begin{tabular}{c}	Objętość\\ \mbox{[$cm^3$]}  \end{tabular} & 
				\begin{tabular}{c}	Gęstość \\ $ \left [\frac{kg}{m^3} \right ] $ \end{tabular}  \\
				\hline
				Aluminium (1)& 0.024 & 44 & 0.196 & 8.639  & 2777.97\\
				\hline
				Aluminium (2) & 0.030 & 55.7 & 0.196 & 10.937  & 2743.05\\
				\hline
				Mosiądz (1) & 0.074 & 31.3 & 0.154  & 4.805 & 7492.50\\
				\hline
				Mosiądz (2)& 0.174 & 22.2 & 0.273  & 8.557 & 8647.53\\
				\hline
				Stal & 0.036 & 0.311  & 0.935 & 20.757 & 8382.71\\
				\hline
				Miedź & 0.066 & 0.440  & 0.189 & 7.279 & 9067.20\\
				\hline
			\end{tabular}
		\end{adjustwidth}
	\end{figure}
	
	\begin{figure}[!htb]
		\begin{adjustwidth}{-1cm}{}
			\def\arraystretch{1.3}
			\centering
			\begin{tabular}{|c|c|c|c|}
				\cline{1-2}
				\multicolumn{2}{|l|}{\begin{tabular}{c} Aluminium (koło) $l = 1 [m]$\end{tabular}} & \multicolumn{2}{c}{}\\
				\hline
				\begin{tabular}{c} Harmoniczna  \end{tabular} & \begin{tabular}{c} Częstotliwość $f$ \\ \mbox{[HZ]}  \end{tabular}  & 
				\begin{tabular}{c}	Długość fali $ \lambda$ \\ \mbox{[m]}  \end{tabular} &
				\begin{tabular}{c} Prędkość fali $\upsilon$ \\ \mbox{[m/s]}  \end{tabular}  \\ 
				\hline
				1 & 93.75 & 2 & 187.50 \\[2ex]
				\hline
				2  & 187.50 & 1 & 187.50 \\[2ex]
				\hline
				3 & 281.25  & 0.67 & 187.50 \\[2ex]
				\hline
				4 & 375.00 & 0,50 & 187.50 \\[2ex]
				\hline
				5 & 468.75 & 0,40 & 187.50 \\[2ex]
				\hline
				6 & 562.50 & 0,33 & 187.50 \\[2ex]
				\hline
				
		\end{tabular}
			$$v=187.50 \left [\frac{m}{s} \right ] $$
			$$E=0.0242 \left [GPa \right ] $$
		\end{adjustwidth}
	\end{figure}
	
		\begin{figure}[!htb]
		\begin{adjustwidth}{-1cm}{}
			\def\arraystretch{1.3}
			\centering
			\begin{tabular}{|c|c|c|c|}
				\cline{1-2}
				\multicolumn{2}{|l|}{\begin{tabular}{c} Mosiądz  (koło) $l = 1 [m]$\end{tabular}} & \multicolumn{2}{c}{}\\
				\hline
				\begin{tabular}{c} Harmoniczna  \end{tabular} & \begin{tabular}{c} Częstotliwość $f$ \\ \mbox{[HZ]}  \end{tabular}  & 
				\begin{tabular}{c}	Długość fali $ \lambda$ \\ \mbox{[m]}  \end{tabular} &
				\begin{tabular}{c} Prędkość fali $\upsilon$ \\ \mbox{[m/s]}  \end{tabular}  \\ 
				\hline
				1 & 93.75 & 2 & 187.50 \\[2ex]
				\hline
				2  & 187.50 & 1 & 187.50 \\[2ex]
				\hline
				3 & 281.25  & 0.67 & 187.50 \\[2ex]
				\hline
				4 & 375.00 & 0,50 & 187.50 \\[2ex]
				\hline
				5 & 468.75 & 0,40 & 187.50 \\[2ex]
				\hline
				6 & 562.50 & 0,33 & 187.50 \\[2ex]
				\hline
				
		\end{tabular}
			$$v=187.50 \left [\frac{m}{s} \right ] $$
			$$E=0.299 \left [GPa \right ] $$
		\end{adjustwidth}
	\end{figure}
	
			\begin{figure}[!htb]
		\begin{adjustwidth}{-1cm}{}
			\def\arraystretch{1.3}
			\centering
			\begin{tabular}{|c|c|c|c|}
				\cline{1-2}
				\multicolumn{2}{|l|}{\begin{tabular}{c} Mosiądz (okrąg) $l = 1 [m]$\end{tabular}} & \multicolumn{2}{c}{}\\
				\hline
				\begin{tabular}{c} Harmoniczna  \end{tabular} & \begin{tabular}{c} Częstotliwość $f$ \\ \mbox{[HZ]}  \end{tabular}  & 
				\begin{tabular}{c}	Długość fali $ \lambda$ \\ \mbox{[m]}  \end{tabular} &
				\begin{tabular}{c} Prędkość fali $\upsilon$ \\ \mbox{[m/s]}  \end{tabular}  \\ 
				\hline
				1 & 93.75 & 2 & 187.50 \\[2ex]
				\hline
				2  & 187.50 & 1 & 187.50 \\[2ex]
				\hline
				3 & 281.25  & 0.67 & 187.50 \\[2ex]
				\hline
				4 & 375.00 & 0,50 & 187.50 \\[2ex]
				\hline
				5 & 468.75 & 0,40 & 187.50 \\[2ex]
				\hline
				6 & 562.50 & 0,33 & 187.50 \\[2ex]
				\hline
				
		\end{tabular}
			$$v=187.50 \left [\frac{m}{s} \right ] $$
			$$E=0.299 \left [GPa \right ] $$
		\end{adjustwidth}
	\end{figure}
	
	\begin{figure}[!htb]
		\begin{adjustwidth}{-1cm}{}
			\def\arraystretch{1.3}
			\centering
			\begin{tabular}{|c|c|c|c|}
				\cline{1-2}
				\multicolumn{2}{|l|}{\begin{tabular}{c} Stal  (koło)  $l = 1,8 [m]$\end{tabular}} & \multicolumn{2}{c}{}\\
				\hline
				\begin{tabular}{c} Harmoniczna  \end{tabular} & \begin{tabular}{c} Częstotliwość $f$ \\ \mbox{[HZ]}  \end{tabular}  & 
				\begin{tabular}{c}	Długość fali $ \lambda$ \\ \mbox{[m]}  \end{tabular} &
				\begin{tabular}{c} Prędkość fali $\upsilon$ \\ \mbox{[m/s]}  \end{tabular}  \\ 
				\hline
				1 & 93.75 & 3,6 & 337.50 \\[2ex]
				\hline
				2  & 187.50 & 1,8 & 337.50 \\[2ex]
				\hline
				3 & 281.25  & 1,2 & 337.50 \\[2ex]
				\hline
				4 & 375.00 & 0,9 & 337.50 \\[2ex]
				\hline
				5 & 468.75 & 0,72 & 337.50 \\[2ex]
				\hline
				6 & 562.50 & 0,6 & 337.50 \\[2ex]
				\hline
				
			\end{tabular}
		\end{adjustwidth}
			$$v=337.50 \left [\frac{m}{s} \right ] $$ 
			$$E=0.853 \left [GPa \right ] $$
	\end{figure}
	
	\begin{figure}[!htb]
		\begin{adjustwidth}{-1cm}{}
			\def\arraystretch{1.3}
			\centering
			\begin{tabular}{|c|c|c|c|}
				\cline{1-2}
				\multicolumn{2}{|l|}{\begin{tabular}{c} Stal (prostokąt) $l = 1,8 [m]$\end{tabular}} & \multicolumn{2}{c}{}\\
				\hline
				\begin{tabular}{c} Harmoniczna  \end{tabular} & \begin{tabular}{c} Częstotliwość $f$ \\ \mbox{[HZ]}  \end{tabular}  & 
				\begin{tabular}{c}	Długość fali $ \lambda$ \\ \mbox{[m]}  \end{tabular} &
				\begin{tabular}{c} Prędkość fali $\upsilon$ \\ \mbox{[m/s]}  \end{tabular}  \\ 
				\hline
				1 & 93.75 & 3,6 & 337.50 \\[2ex]
				\hline
				2  & 187.50 & 1,8 & 337.50 \\[2ex]
				\hline
				3 & 281.25  & 1,2 & 337.50 \\[2ex]
				\hline
				4 & 375.00 & 0,9 & 337.50 \\[2ex]
				\hline
				5 & 468.75 & 0,72 & 337.50 \\[2ex]
				\hline
				6 & 562.50 & 0,6 & 337.50 \\[2ex]
				\hline
				
			\end{tabular}
		\end{adjustwidth}
			$$v=337.50 \left [\frac{m}{s} \right ] $$ 
			$$E=0.853 \left [GPa \right ] $$
	\end{figure}
	
		\begin{figure}[!htb]
		\begin{adjustwidth}{-1cm}{}
			\def\arraystretch{1.3}
			\centering
			\begin{tabular}{|c|c|c|c|}
				\cline{1-2}
				\multicolumn{2}{|l|}{\begin{tabular}{c} Miedź  (koło) $l = 1,8 [m]$\end{tabular}} & \multicolumn{2}{c}{}\\
				\hline
				\begin{tabular}{c} Harmoniczna  \end{tabular} & \begin{tabular}{c} Częstotliwość $f$ \\ \mbox{[HZ]}  \end{tabular}  & 
				\begin{tabular}{c}	Długość fali $ \lambda$ \\ \mbox{[m]}  \end{tabular} &
				\begin{tabular}{c} Prędkość fali $\upsilon$ \\ \mbox{[m/s]}  \end{tabular}  \\ 
				\hline
				1 & 93.75 & 3,6 & 337.50 \\[2ex]
				\hline
				2  & 187.50 & 1,8 & 337.50 \\[2ex]
				\hline
				3 & 281.25  & 1,2 & 337.50 \\[2ex]
				\hline
				4 & 375.00 & 0,9 & 337.50 \\[2ex]
				\hline
				5 & 468.75 & 0,72 & 337.50 \\[2ex]
				\hline
				6 & 562.50 & 0,6 & 337.50 \\[2ex]
				\hline
				
			\end{tabular}
		\end{adjustwidth}
			$$v=337.50 \left [\frac{m}{s} \right ] $$ 
			$$E=1.033 \left [GPa \right ] $$
	\end{figure}
	
	\clearpage
	\section{Opracowanie wyników}
	Błędy pomiarów:\\
	Długości pręta: $u(l)=0.001m$\\
	Promienia lub boku podstawy: $u(r)=0.00005m$\\
	Masy próbki: $u(m)=0.001kg$\\
	Częstotliwości: $u(f)=0.25Hz$\\
	
	Niepewność gęstości:
	\[
	u(\rho)=\sqrt{\bigg(\frac{\partial \rho}{\partial m}u(m)\bigg)^2+\bigg(\frac{\partial \rho}{\partial l}u(l)\bigg)^2+\bigg(\frac{\partial \rho}{\partial r}u(r)\bigg)^2} = \sqrt{\bigg(\frac{1}{l\pi r^2}u(m)\bigg)^2+\bigg(\frac{-m}{l^2 \pi r^2}u(l)\bigg)^2+\bigg(\frac{-2m}{l\pi r^3}u(r)\bigg)^2}
	\]
	
	Niepewność długości fali:
	$$ u(\lambda)=\sqrt{\bigg(\frac{2}{n}u(l)\bigg)^2}$$
	
	Niepewność prędkości fali:
	$$ u(v)=\sqrt{\bigg(\frac{\partial v}{\partial f}u(f)\bigg)^2+\bigg(\frac{\partial v}{\partial \lambda}u(\lambda)\bigg)^2}=\sqrt{\bigg(\lambda u(f)\bigg)^2+\bigg(f u(\lambda)\bigg)^2}$$
	
	Niepewność modułu Younga:
	$$ u(E)=\sqrt{\bigg(\frac{\partial E}{\partial \rho}u(\rho)\bigg)^2+\bigg(\frac{\partial E}{\partial v}u(v)\bigg)^2} =
	\sqrt{\bigg(v^2 u(\rho)\bigg)^2+\bigg(2 \rho v  \cdot{}u(v)\bigg)^2}$$
	
\begin{figure}[!htb]
	\caption{Tabela niepewności dla gęstości}

		\def\arraystretch{1.3}
		\centering
\begin{tabular}{|c|c|c|c|c|c|}
	\hline
	\begin{tabular}{c} Materiał  \end{tabular} & \begin{tabular}{c} Niepewność \\ złożona \\gęstości\end{tabular}  & 
	\begin{tabular}{c} Niepewność\\ rozszerzona\\ $k=2 $    \end{tabular}
	&
	\begin{tabular}{c} Gęstość\\ obliczona  \\ $\left [ \frac{kg}{m^3} \right ] $\end{tabular}
	&
	\begin{tabular}{c}	Gęstość \\tabelaryczna\\ $\left [ \frac{kg}{m^3} \right ] $ \end{tabular} &
	\begin{tabular}{c} Czy w przedziale \\ niepewności  \end{tabular}  \\ 
	\hline
	Aluminium (1)   & 160.58    & 321.16    & 2777.97  & 2720 & TAK \\[0.3ex]
	\hline
	Aluminium (2)   & 142.92    & 185.84    & 2743.05  & 2720 & TAK \\[0.3ex]
	\hline
	Mosiądz (1)     & 47.39    & 94.78   & 8647.53  & 8400–8700 & TAK \\[0.3ex]
	\hline
	Mosiądz (2)     & 78.47     & 156.94     & 8382.71  & 8400–8700 & TAK \\[0.3ex]
	\hline
	Stal            & 108.62    & 217.24    & 7492.50  & 7500–7900 & TAK\\[0.3ex]
	\hline
	Miedź           & 126.91    & 253.82    & 9067.20  & 8933 & TAK\\[0.3ex]
	\hline
	
\end{tabular}
\end{figure}

\begin{figure}[!htb]
	    \caption{Tabela niepewności dla prędkości dźwięku}
		\def\arraystretch{1.3}
		\centering
    \begin{tabular}{|c|c|c|c|c|}
    
    	\hline
    	\begin{tabular}{c} Materiał  \end{tabular} & \begin{tabular}{c} Niepewność złożona \\prędkości dźwięku\end{tabular}  & 
    	\begin{tabular}{c} Niepewność\\rozszerzona $k=2$\end{tabular}
    	&
    	\begin{tabular}{c} Prędkość \\obliczona  \\ $\left [ \frac{m}{s} \right ] $\end{tabular}
    	&
    	\begin{tabular}{c}	Prędkość \\tabelaryczna\\ $\left [ \frac{m}{s} \right ] $ \end{tabular}\\ 
    	\hline
    	Aluminium 1m  (koło)         & 50     & 100    & 187.50  & 6300 \\[0.3ex]
    	\hline
    	Mosiądz 1m  (koło)           & 50     & 100    & 187.50  & 3500 \\[0.3ex]
    	\hline
    	Mosiądz 1m (okrąg)    & 50     & 100    & 187.50  & 3500 \\[0.3ex]
    	\hline
    	Stal 1.8m (koło)              & 90     & 180    & 337.50  & 5100-6000 \\[0.3ex]
    	\hline
    	Stal 1.8m (prostokąt)         & 90     & 180    & 337.50  & 5100-6000 \\[0.3ex]
    	\hline
    	Miedź 1.8m (okrąg)              & 90     & 180    & 337.50  & 3900 \\[0.3ex]
    	\hline
    	
    \end{tabular}
\end{figure}
	
\begin{figure}[!htb]
        \caption{Tabela niepewności dla modułu Younga}

		\def\arraystretch{1.3}
		\centering
    \begin{tabular}{|c|c|c|c|c|}
    	\hline
    	\begin{tabular}{c} Materiał  \end{tabular} & \begin{tabular}{c} Niepewność złożona \\modułu Younga\end{tabular}  & 
    	\begin{tabular}{c} Niepewność\\rozszerzona\\$k=2$\end{tabular}
    	&
    	\begin{tabular}{c} Moduł\\ obliczony  \\ $[ GPa ] $\end{tabular}
    	&
    	\begin{tabular}{c}	Moduł Younga \\wartość tablicowa\\ $[ GPa ] $ \end{tabular}\\ 
    	\hline
    	Aluminium 1m  (koło)            & $0.0523\cdot 10^9$    & 100    & 0.0242  & 69 \\[0.3ex]
    	\hline
    	Mosiądz 1m  (koło)              & $0.1597\cdot 10^9$     & 100    & 0.299   & 103-124 \\[0.3ex]
    	\hline
    	Mosiądz 1m (okrąg)              & $0.1597\cdot 10^9$    & 100    & 0.299   & 103-124 \\[0.3ex]
    	\hline
    	Stal 1.8m (koło)                & $0.4553\cdot 10^9$     & 180    & 0.853  & 205-210 \\[0.3ex]
    	\hline
    	Stal 1.8m (prostokąt)           & $0.4553\cdot 10^9$     & 180    & 0.853  & 205-210 \\[0.3ex]
    	\hline
    	Miedź 1.8m (okrąg)              & $0.5510\cdot 10^9$     & 180    & 1.033  & 110-135 \\[0.3ex]
    	\hline
    	
    \end{tabular}
\end{figure}
\pagebreak
	\section{Wnioski}
	
    Wartości, gęstości dla różnych materiałów są zgodne z wartościami tabelarycznymi, dla obliczonej niepewności rozszerzonej. Świadczy to o prawidłowym zmierzeniu przekrojów, długości, oraz wag próbek materiałów.
    
    
    Wyniki pomiaru częstotliwości przy użyciu programu Zelscope są identyczne co świadczy o nieprawidłowym działania bądź niewłaściwym użycia oprogramowania i/ lub aparatury pomiarowych.
    
    
    Błędy pomiaru częstotliwości przeniosły się dalej co spowodowało zakłamanie pozostałych wyników

	
\end{document}