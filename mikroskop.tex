\documentclass{article}

\usepackage{polski}
\usepackage[utf8]{inputenc}
\usepackage{graphicx}
\usepackage{float}
\usepackage[margin=1in]{geometry}
\usepackage{graphicx}
\usepackage{amsmath}
\usepackage{mathtools}
\usepackage{amssymb}
\usepackage{multirow}
\usepackage{changepage}
\usepackage[export]{adjustbox}
\usepackage{wrapfig}
\usepackage{caption}


\title{Sprawozdanie}

\begin{document}
	\begin{center}
		\bgroup
		\def\arraystretch{1.5}
		\begin{tabular}{|c|c|c|c|c|c|}
			\hline
			EAIiIB & \multicolumn{2}{|c|}{\begin{tabular}{@{}c@{}}Autor 1: Rafał Mazur \\Autor 2: Jakub Ficoń\end{tabular}} & Rok II & Grupa 5 & Zespół 3 \\
			\hline
			\multicolumn{3}{|c|}{\begin{tabular}{c}Temat: \\ Współczynnik załamania światła dla ciał stałych \end{tabular}} & 
			\multicolumn{3}{|c|}{\begin{tabular}{c}Numer ćwiczenia: \\ 51 \end{tabular}} \\
			\hline
			Data wykonania & Data oddania & Zwrot do poprawki & Data oddania & Data zaliczenia & Ocena \\[8ex]
			\hline
		\end{tabular}
		\egroup
	\end{center}  
	
	\section{Cel ćwiczenia}
	Wyznaczanie współczynnika załamania światła dla płytki szklanej i pleksiglasowej metodą pomiaru grubości pozornej płytki przy pomocy mikroskopu.
	
	\section{Wstęp teoretyczny}
	    W załącznikach na końcu sprawozdania
	
	\setcounter{section}{3}
	\newpage
	\setcounter{page}{5}
	\section{Opracowanie wyników pomiarów}

	\noindent Wzór na współczynnik załamania
	\begin{equation}
	    n=\frac{d}{h}
	\end{equation}

	\noindent Gdzie: n-współczynnik załamania, d- grubość rzeczywista, h- grubość pozorna.
	
	\subsection{Płytka z pleksiglasu:}
	Grubość rzeczywista d=3.84[mm]
	
	\noindent Niepewność u(d)=0.01[mm]
	
	\noindent Średnia grubość pozorna h=2.56[mm]
	
	\noindent Niepewność u(h)=0.016[mm]
	
	\noindent Korzystając ze wzoru (1) wyznaczam wartość współczynnika załamania dla płytki z pleksiglasu:
	
	\[
	    n=\frac{3.84}{2.56}=1.501
	\]
	
	\subsection{Płytka szklana:}
	Grubość rzeczywista d=3.76[mm]
	
	\noindent Niepewność u(d)=0.01[mm]
	
	\noindent Średnia grubość pozorna h=2.47[mm]
	
	\noindent Niepewność u(h)=0.020[mm]
	
	\noindent Korzystając ze wzoru (1) wyznaczam wartość współczynnika załamania dla płytki szklanej:
	
	\[
	    n=\frac{3.76}{2.47}=1.520
	\]
	
	\subsection{Płytka szklana z filtrem niebieskim:}
	Grubość rzeczywista d=3.76[mm]
	
	\noindent Niepewność u(d)=0.01[mm]
	
	\noindent Średnia grubość pozorna h=2.51[mm]
	
	\noindent Niepewność u(h)=0.016[mm]
	
	\noindent Korzystając ze wzoru (1) wyznaczam wartość współczynnika załamania dla płytki szklanej:
	
	\[
	    n=\frac{3.76}{2.51}=1.496
	\]
	
	\subsection{Płytka szklana z filtrem zielonym:}
	Grubość rzeczywista d=3.76[mm]
	
	\noindent Niepewność u(d)=0.01[mm]
	
	\noindent Średnia grubość pozorna h=2.49[mm]
	
	\noindent Niepewność u(h)=0.016[mm]
	
	\noindent Korzystając ze wzoru (1) wyznaczam wartość współczynnika załamania dla płytki szklanej:
	
	\[
	    n=\frac{3.76}{2.49}=1.510
	\]
	
	\subsection{Płytka szklana z filtrem czerwonym:}
	Grubość rzeczywista d=3.76[mm]
	
	\noindent Niepewność u(d)=0.01[mm]
	
	\noindent Średnia grubość pozorna h=2.50[mm]
	
	\noindent Niepewność u(h)=0.018[mm]
	
	\noindent Korzystając ze wzoru (1) wyznaczam wartość współczynnika załamania dla płytki szklanej:
	
	\[
	    n=\frac{3.76}{2.50}=1.502
	\]
	
	\section{Obliczenie niepewności:}
	
	Wzór na niepewność złożoną współczynnika załamania:
	\begin{equation}
	    u(n)=\sqrt{\left[\frac{1}{h}u(d)\right]^{2}+\left[\frac{-d}{h^{2}}u(h)\right]^{2}}
	\end{equation}

	\begin{table}[!htb]
	\centering
    \begin{tabular}{|c|c|c|c|}
    \hline
     & $u(d)$ & $u(h)$ & $u(n)$ \\ \hline
     Pleksiglas                 & 0.01 & 0.016 & 0.011\\ \hline
     Szkło                      & 0.01 & 0.020 & 0.013 \\ \hline
     Szkło z filtrem niebieskim & 0.01 & 0.016 & 0.010 \\ \hline
     Szkło z filtrem zielonym   & 0.01 & 0.016 & 0.011 \\ \hline
     Szkło z filtrem czerwonym  & 0.01 & 0.018 & 0.012 \\ \hline

    \end{tabular}
        \caption{Tabela przed zaokrągleniem}
    \end{table}

    \section{Zestawienie wyników}
    
    \begin{table}[!htb]
	\centering
    \begin{tabular}{|c|c|c|c|c|c|}
    \hline
     Materiał & n zmierzone & n tablicowe & $u(n)$ & $u(n)$ rozszerzone & Równe? \\ \hline
     Pleksiglas & 1.501 & 1.489 & 0.011 & 0.022 & TAK    \\ \hline
     Szkło                      & 1.520 & 1.500 & 0.013 & 0.026 & TAK\\ \hline
     Szkło z filtrem niebieskim & 1.496 & 1.500 & 0.010 & 0.020 & TAK\\ \hline
     Szkło z filtrem zielonym & 1.510 & 1.500 & 0.011 & 0.022 & TAK\\ \hline
     Szkło z filtrem czerwonym & 1.502 & 1.500 & 0.012 & 0.024 & TAK\\ \hline


    \end{tabular}
        \caption{Zestawienie wyników}
    \end{table}
    
    \section{Wnioski}
    Z wykonanych obliczeń wynika, że metoda wyznaczania współczynnika załamania metodą pomiaru grubości pozornej płytki przy pomocy mikoskopu jest dobrym sposobem wyznaczania tej wartości, gdyż wszystkie obliczone wartości pokrywają się z wartościami tablicowymi w granicach niepewności rozszerzonej. Wykorzystanie kolorowych filtrów światła pokazało, że kolor światła nie wpływa na współczynnik załamania, nie można jednak stwierdzić tego z pewnością gdyż różnice pomiędzy pomiarami są zbyt małe. Niewielkie rozbieżności w wynikach mogą wynikać z subiektywności postrzegania obrazu za ostry. Nie można było dokładnie sprawdzić zgodności szkła z wartością tablicową gdyż nie można było stwierdzić rodzaju szkła więc przyjęta została wartość średnia dla tego materiału czyli $n=1.500$

	
\end{document}